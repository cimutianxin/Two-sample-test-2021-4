\documentclass[UTF-8]{ctexart}
\usepackage[colorlinks,linkcolor=blue]{hyperref}
\usepackage[left=1.25in,right=1.25in,top=1in,bottom=1in]{geometry}
\usepackage{graphicx}
\usepackage{listings}
\lstset{
columns=fullflexiblem,
breaklines=true,
frame=none,
backgroundcolor=\color[RGB]{245,245,244},
basicstyle=\tt,
extendedchars=false 
}
\begin{document}
\author{Singtren}
\title{LaTeX即时预览on Windows}
\maketitle
LaTeX对于学数学或物理的同学应该是必会的技能了。写论文,做笔记,交作业,都可以用LaTeX完成。但它并不是界面友好的系统,入门曲线出了名的陡峭。最大的问题就是它不是所见即所得的。

很多人,特别是精通LaTeX的人,都说非所见即所得反而能提高编辑的效率。这一点我到现在还没能理解,或许是我的LaTeX还处于入门水平的原因吧。所以,对我来说,要是我能在写LaTeX代码的时候能即时看到编译后的文档,就能极大地提高编辑效率。当然,这是指文档有很多数学公式的情况下。对于没公式的文档,我并不需要及时预览,或者干脆不用LaTeX.我想,对于大多数LaTeX用户来说会和我有同样的想法。
对于会Linux的同学,写个LaTeX及时预览脚本应该不是问题。

大多数同学应该是Windows用户,而在Windows上没有支持即时预览的LaTeX集成环境。
解决方案有两个。

\begin{itemize}
\item 浏览器做编辑器,MathJax及时渲染
\item SumatraPdf+vim+即时预览插件
\end{itemize}
对于第一种方法,推荐\href{https://zohooo.github.io/jaxedit/}{JaxEditor},可以在线使用,也可以用离线版。
\begin{figure}[]
	\centering
	\includegraphics[width=17cm]{jax.jpg}
	\caption{JaxEditor}
\end{figure}
在没有使用第二种方法之前,JaxEditor我用了很久,公式特别多的文档一般都在JaxEditor上一段一段地编写,然后一段一段复制到WinEdt 编译。这样避免了很多公式语法错误。不然,如果不能预览公式,写了一大段代码,编译报错,那要花很久才能找到错误。

JaxEditor是我非常喜欢的编辑器,因为在浏览器上写公式非常方便,离线版打开直接写,立即能预览,也不用新建文件。并且MathJax 渲染的公式很漂亮。做笔记需要写公式的时候,在JaxEditor写完,然后直接截图粘贴就行。

不过MathJax也有局限性,毕竟使用的是浏览器,只能支持部分LaTeX命令,当然对于只写数学公式的话已经完全够了.还有一个问题是,段落写得有点长时输入会有明显的延迟现象,公式渲染也会延迟。而书写不流畅是我最不能忍受的。没法忍受用一支墨水断断续续的笔书写数学式子。对于输入有能感受到的延迟的文本编辑器我是绝对不会用的。所以JaxEdit我一般是写一段,清一段。

当遇到写较长论文的情况时,每一段在JaxEdit上写完再复制效率将是非常低的。这时需要的是能即时编译的系统。找了很久,终于还是被我找到了可行的方案。那就是\href{https://github.com/xuhdev/vim-latex-live-preview}{vim-latex-live-preview}

其实在一年前我就发现了这个vim插件,但我看到作者说这个插件只支持Linux系统。所以当时我就没去试(当时差点就冲动地去换Linux 系统)。一年后再次看到这个插件,这是已经会用GitHub了。就把这个repository clone下来。看了一下代码,也就不到200行,代码兼容性也很好,Windows 上应该也能跑起来的。看了一下项目的issue,原来作者说不支持Windows的原因是作者以为Windows上没有及时预览的pdf 阅读器。因为像Adobe Reader 的阅读器打开文件时是不允许其他软件改写这个文件的。但是Windows上有SumatraPdf,能有支持及时预览。那么这个问题就解决了。
做法是将\verb!vim-latex-live-preview/plugin/latexlivepreview.vim!复制到\verb!Vim/vimfiles/plugin!.
在\verb!Vim/vimrc!中添加
\begin{lstlisting}
autocmd Filetype tex setl updatetime=1000
let g:Tex_MultipleCompileFormats='pdf'
let g:livepreview_previewer = 'SumatraPDF -reuse-instance -inverse-search "gvim -c \":RemoteOpen +\%l \%f\""'
\end{lstlisting}
当然,前提是安装了Tex-Suit和Python.因为vim-latex-preview是用Python写的.至于Tex-Suit的配置,这里就不讲了。

做完后,用vim打开一个TeX文档,输入\verb!:LLPStartPreview!就能用SumatraPDF进行及时预览。经测试,在写文档的时候实时编译并不影响输入响应时间.打字依旧那么流畅。不过要是按住hjkl让光标持续快速移动时,还是会出现卡顿现象。并且CPU和磁盘使用率会很高.


其实,有了支持及时更新的阅读器SumatraPDF配合vim,也不怎么需要及时编译及时预览了。只要把SumatraPDF放一边。写玩一段按\verb!\ll! 编译一下,SumatraPDF自动会刷新文档。

总之在Windows上写LaTeX最好的搭配是vim+TeX-Suit+SumatraPDF

及时预览效果见图
\begin{figure}[]
	\centering
	\includegraphics[width=17cm]{livepreview.jpg}
	\caption{使用插件及时预览}
\end{figure}

\end{document}
