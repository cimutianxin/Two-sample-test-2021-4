\documentclass[12pt]{article}
%sometimes use book, report...
\usepackage{ctex}
\usepackage{graphicx}
\usepackage{listings}
\usepackage{xcolor}
\usepackage{keyval}
\usepackage{enumerate}
\usepackage{amsfonts}
\usepackage{amsmath}
\usepackage{amsthm}
\usepackage{pifont}
\usepackage{hyperref}
\usepackage{fontspec}
\usepackage[numbers, sort&compress]{natbib}
\usepackage{multicol}
\usepackage{float}
\usepackage{appendix}
\usepackage{subfigure}
\usepackage{hyperref}
\usepackage{booktabs}

%超链接网址
\hypersetup{
    colorlinks=true,
    linkcolor=blue,
    filecolor=blue,      
    urlcolor=blue,
    citecolor=cyan,
}

%编号超链接及样式
\hypersetup{hypertex=false, 
            colorlinks=true, 
            linkcolor=black, 
            anchorcolor=black, 
            citecolor=black}

%图片按章节编码
\numberwithin{figure}{section}

%section从0编号
\setcounter{section}{-1}

%引用cite角标\upcite
\newcommand{\upcite}[1]{\textsuperscript{\textsuperscript{\cite{#1}}}}

%emoji
\newfontface\EmojiFont{Segoe UI Emoji}

%边距
\usepackage{geometry}   %设置页边距的宏包
\usepackage{titlesec}   %设置页眉页脚的宏包
\geometry{left=2.5cm, right=2.5cm, top=1cm, bottom=2.5cm}  %设置 上、左、下、右 页边距

%代码格式
\definecolor{winered}{rgb}{0.5, 0, 0}
\definecolor{lightgrey}{rgb}{0.95, 0.95, 0.95}
\definecolor{commentcolor}{RGB}{0, 100, 0}
\definecolor{frenchplum}{RGB}{190, 20, 83}
\definecolor{codepurple}{rgb}{0.58, 0, 0.82}
\lstset{flexiblecolumns, %%numbers=left, numberstyle=\footnotesize,
 breaklines=true, keywordstyle=\color{winered}, 
    frame=tlbr, framesep=4pt, framerule=0pt, commentstyle=\color{commentcolor}, 
    stringstyle=\color{codepurple}, tabsize=2, backgroundcolor=\color{lightgrey}
}

%图片路径
\graphicspath{{image/}}

\title{几个统计检验方法的实现与评价}
\author{统计81柴思楚2183310799}

\begin{document}
\maketitle

\section{引言}
实现 two sample independent test 检验 categorical 数据
中两个变量的关系,比较 chi-square test 和 fisher exact test,
在模拟数据和一个实际生物数据上评价这两种方法。

实现 two sample dependent test 检验 categorical 数据中
两个变量的关系,在模拟数据和一个实际生物数据上评价该方法。(如
McNemar test)

实现 Wilcoxon rank sum test,在模拟数据和实际数据上
和 t-test 比较, 评价这两种方法。 (可以只考虑数据不打结的情况。)
注:样本量大的时候可以用 normal approximation 或 permutation
test,样本量小时可以穷举 H0 中统计量的分布。


\section{two sample independent test}
这一节中将实现和评价chi-square test 与 fisher exact test。

首先考虑$2\times2$的数据,即两个变量之间的关系。

\subsection{模拟数据}
\subsubsection*{实验1:两个班级考试及格率比较}
假设有200个新生入学,他们被随机分为两组,分别由A,B两位老师教授同一门课,学期结束后参加同一场考试,记录成绩及格与否,以下是结果。(结果通过随机数生成,假设A老师有90\% 的通过率,B老师有80\% 的通过率)
\begin{table}[H]
    \centering
    \begin{tabular}{|l|l|l|l|}
    \hline
          & teacher A & teacher B & Total \\ \hline
    Pass  & 92        & 81        & 173   \\ \hline
    Fail  & 8         & 19        & 27    \\ \hline
    Total & 100       & 100       & 200   \\ \hline
    \end{tabular}
\end{table}

原假设为学生及格率与教师无关,即教师A与教师B的班级及格率相同。
\begin{equation*}
    \begin{aligned}
        H_0:\mu_1&=\mu_2\\
        H_1:\mu_1&\neq\mu_2
    \end{aligned}
\end{equation*}

利用chi-square test,得到$\chi^{2}=4.28, p=.039<.05$, 故拒绝原假设,因此学生及格率与教师有关。

利用fisher exact test,得到$p=.037<.05$, 故拒绝原假设,因此学生及格率与教师有关。

可以看出,在较大样本量的情况下,chi-square test 和 fisher exact test,得到了相同的结果。


\subsubsection*{实验2:性别与现在是否在学习的关系}
假设有一组学生样本首先被分为男女两组,并根据是否正在学习进行统计,以下是结果。

\begin{table}[H]
    \centering
    \begin{tabular}{|l|l|l|l|}
    \hline
                 & Men & Women & Total \\ \hline
    Studying     & 1   & 9     & 10    \\ \hline
    Not-studying & 11  & 3     & 14    \\ \hline
    Total        & 12  & 12    & 24    \\ \hline
    \end{tabular}
\end{table}

原假设为男生和女生同等可能的正在学习。
\begin{equation*}
    \begin{aligned}
        H_0:\mu_1&=\mu_2\\
        H_1:\mu_1&<\mu_2
    \end{aligned}
\end{equation*}

利用fisher exact test,得到$p=.0028<.05$, 故拒绝原假设,因此男生学习的可能性较女生更低。

利用chi-square test,得到$\chi^{2}=8.4, p=.0038<.05$, 故拒绝原假设,因此男生学习的可能性较女生更低。
但事实上,在进行小样本分析时,有可能会得到小于5的理论值,从而导致chi-square test不准,例如[[6,9],[6,3]]的情况。
实际得到的p-值结果,在多次测试中,有可能出现两种测试p-value相差较大的情况,从而导致结论不同,因此,在小样本比较中,应当采用fisher exact test。


\subsection{实际生物数据}

\begin{table}[H]
    \centering
    \begin{tabular}{lrrr}
    \hline
    $\text{影响因素}$ & \multicolumn{1}{l}{$\text{例数}$} & \multicolumn{1}{l}{$\text{复发}$} & \multicolumn{1}{l}{$\text{未复发}$} \\ \hline
    \multicolumn{4}{l}{$\text{性别}$}                                                                                      \\
    $\text{男}$    & 79                              & 9                               & 70                               \\
    $\text{女}$    & 75                              & 7                               & 68                               \\ \hline
    \end{tabular}
\end{table}












\section{项目地址}
\url{https://github.com/cimutianxin/cluster-2021-4}

\end{document}


%交大封面
\begin{titlepage}
    ~\\[15pt]
    \begin{figure*}[h]
        \centering
        \includegraphics[width=0.8\textwidth]{xjtu.png} 
    \end{figure*}
    \vspace{1.5cm}
    \begin{center}
        {\Huge 题目:《论语》智慧与现代文明书评}
    \end{center}
    \vspace{2.5cm}
    \begin{enumerate}[]
        \item \hspace{4.3cm}\Large 课程:能源与人类文明
        \item \hspace{4.3cm}\Large 班级:统计81
        \item \hspace{4.3cm}\Large 姓名:柴思楚
        \item \hspace{4.3cm}\Large 学号:2183310799
    \end{enumerate}  
    \vspace{3.5cm}
    \hspace{10cm}\large 2020年7月
\end{titlepage}

%图片
\begin{figure}
    \centering
    \includegraphics[width=0.8\textwidth]{1.png} 
    \caption{qwer}\label{123}
\end{figure}

%代码
\begin{lstlisting}[language=C++]
    // demo
    #include <stdio.h>
    int main () {
        printf("Hello world!\n");
    }
\end{lstlisting}  

%引用
\begin{quotation}
    子曰:“参乎!吾道一以贯之.”

    曾子曰:“唯.”

    子出.门人问曰:“何谓也?”
    
    曾子曰:“夫子之道, 忠恕而已矣.”(《论语·里仁》)
\end{quotation}